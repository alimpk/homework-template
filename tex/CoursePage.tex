\newpage
\FirstPageStyle
{
\settextfont[Scale=1.2]{IRZar}
    \begin{center}
		\bfseries 
نکاتی در رابطه با نوع تمرین
    \end{center}
	
	\begin{itemize}[noitemsep]
		\setlength\itemsep{0em}
            \item 	
                \textbf{مرور و تثبیت مفاهیم:}
پرسش‌های این بخش جهت مرور و یادآوری مفاهیم درسی آورده شده است و با مطالعه مفاهیم درسی گفته‌شده در کلاس درس و اسلایدهای درس خواهید توانست به آن‌ها پاسخ دهید. پاسخ آن‌ها مورد ارزیابی قرار نخواهد گرفت، لذا نیازی به ارسال پاسخ آن‌ها نیست.

            \item
            \textbf{تحلیل و طراحی مدار:}
پرسش‌های این بخش جهت درک عمیق مفاهیم درسی و افزایش قدرت تحلیل و طراحی سیستم‌های دیجیتال آورده شده است. پاسخ به آن‌ها الزامی بوده و مورد ارزیابی قرار خواهد گرفت.
            \item
            \textbf{توصیف و پیاده‌سازی:}
 پرسش‌های این بخش جهت افزایش مهارت شما در پیاده‌سازی مدارهای دیجیتال، بررسی درستی عملکرد آن و استفاده از ابزارهای طراحی آورده شده است. پاسخ به آن‌ها الزامی بوده و مورد ارزیابی قرار خواهد گرفت.
		
            \item
            \textbf{طراحی و پیاده‌سازی سامانه پیشرفته:}
پرسش‌های این بخش ممکن است کمی پیچیده‌تر و دشوارتر از سایر بخش‌ها باشد. الزامی یا اختیاری بودن آن‌ها در صورت پرسش ذکر شده است. 
    \end{itemize}

    \begin{center}
    \bfseries 
نکاتی در رابطه با نحوه ارسال تمرین
    \end{center}

    \begin{itemize}[noitemsep]
		\setlength\itemsep{0em}
		\item
ارسال تمرینات به‌صورت الکترونیکی و از طریق سامانه دروس  خواهد بود. فایل ارسالی شما یک فایل فشرده zip  با نام \verb;SID_HS.zip; است که \verb;SID; شماره دانشجویی و \verb;HS; شماره سری  تمرین است. یک قالب آماده در سامانه دروس قرار داده شده است تا پاسخ تمرین را در قالب تعیین‌شده بنویسید.
		\item
پرسش‌هایی که پاسخ آن‌ها ماهیت تشریحی و تحلیلی دارد را به‌صورت  تایپی یا دستی نوشته و در قالب فایل PDF تحویل دهید. برای پرسش‌هایی که ماهیت کدنویسی و پیاده‌سازی دارند یک پوشه با نام ‌ \verb;src; در سامانه قرار داده شده است. آن را تکمیل نموده و ارسال نمایید.	
        \item
زمان تحویل هر سری از تمرین‌ها مشخص بوده و پاسخ تمرین پس از موعد مقررشده در سامانه درس قرار داده خواهد شد؛ لذا امکان تغییر آن وجود ندارد. در حل تمرینات، می‌توانید به‌صورت دوتایی یا چندتایی با یک‌دیگر همفکری و بحث نمایند ولی هر شخص می‌بایست درنهایت جواب و استدلال خود را به‌صورت انفرادی بنویسد. در صورت شباهت پاسخ، تمامی افراد نمره تمرین را از دست خواهند داد.
    \end{itemize}

چنانچه ابهامی در زمینه تمرینات دارید، می‌توانید اشکالات خود را از طریق پست الکترونیکی زیر با موضوع
    \textbf{\lr{PDS-2020: Subject}}
رفع نمایید.
    \begin{flushleft}
    \href{mailto:ali.mohammadpour@aut.ac.ir}{ali.mohammadpour@aut.ac.ir}\\
علی محمدپور\\
موفق و پیروز باشید!
    \end{flushleft}
}
